%!TEX root = ../thesis.tex
%*******************************************************************************
%*********************************** Chapter X *****************************
%*******************************************************************************

\chapter{Introduction}  %Title of the Chapter

% \ifpdf
%     \graphicspath{{Chapter1/Figs/Raster/}{Chapter1/Figs/PDF/}{Chapter1/Figs/}}
% \else
%     \graphicspath{{Chapter1/Figs/Vector/}{Chapter1/Figs/}}
% \fi



%********************************** %First Section  **************************************
\section{The Cognitive Revolution}


The anonymous author of the eleventh century Hebrew grammar, \textit{Kitāb al-ʿUqūd}, begins 
his text the origin of language:

\begin{quotation}
  Take note that rational beings examined such sounds as the creaking of a door, the noise of 
  a bird, and others. Then they extracted speech sounds from these noises because they needed
  verbal expressions in order to understand each other's intentions, since pointing was not as 
  adequate for this purpose as words. \cite[26]{Vidro2013}
\end{quotation}

From there, humans ``contented themselves with twenty-two basic consonants,'' (\citeauthor[26]{Vidro2013})
deeming that this was enough. Then they made alphabets and vowels.